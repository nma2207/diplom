
\documentclass{lomonosov}

\begin{thesis}  % Сам тезис должен быть полностью помещен внутри окружения thesis


\Title{Восстановление размытых изображений}{{Набиев\,М.\,А.}}

%
%   Команда авторства. Выберете ту, что отвечает вашему тезису, и, если надо, раскомментируйте ее; остальные --- удалите или закомментируйте.
%

% Один автор
\Author{Набиев~Марат~Айратович}{Студент}{Институт ВМиИТ К(П)ФУ}{Казань}{Россия}{nma2207@gmail.com}

% Несколько авторв из одной организации
%\Author{Ильин Александр Владимирович, Шевцова Ирина Геннадьевна}{Математик, ассистент}{Факультет ВМК МГУ имени М.\,В.\,Ломоносова}{Москва}{Россия}{smu@cs.msu.ru, lomonosov@cs.msu.ru}

% Несколько авторов из разных организаций
%\AuthorM{{Иванов~Иван~Иванович}{Петров~Петр~Петрович}}{%
%   {Аспирант, факультет ВМК МГУ имени М.\,В.\,Ломоносова, Москва, Россия}{Младший научный сотрудник, Ленинградский кораблестроительный институт, Ленинград, СССР}}{ivanov@cmc.msu.ru, petrov@cmc.msu.su}

Восстановление размытых изображений - сложная задача в цифровой обработке изображений. Для начала надо рассмотреть математическую модель размытия.
\begin{equation}\label{iline-shevtsova-eq1}
g(x,y) = f(x,y) \otimes h(x,y) + n(x,y) = \sum_{m=0}^{M-1} \sum_{n=0}^{N-1} f(m,n)h(x-m, y-n)+n(x,y)
\end{equation}
Где $g(x,y)$ - размытое изображение, $f(x,y)$ - исходное изображение, $h(x,y)$ - ядро размытия, $n(x,y)$ - шум.

Эта операция называется сверткой. Для восстановления нужно выполнить обратную свертку, только это очень сложный процесс. Здесь рассматриваются некоторые методы решения этой задачи

\paragraph{С известным ядром размытия}\
Рассмотрим методы, предполагающие, что ядро размытия известно

\begin{enumerate}
\item Инверсный фильтр. По теореме о светке, свертка эквивалентна по-элементному перемножению спектров исходного изображения и ядра размытия, тогда спектр приближенного исходного изображения можно выразить так:

\begin{equation}\label{iline-shevtsova-eq1}
\hat{F}(u,v) =\frac{G(u,v)}{H(u,v)}
\end{equation}

%
%Фильтр Винера
%
\item Фильтр Винера. В этом методе минимизируется стандратное отклонение приближенного значения исходного изображения от восстановленного. И следующиее приближение и есть минимальное:
\begin{equation}\label{iline-shevtsova-eq1}
\hat{F}(u,v) =\left(  \frac{H'(u,v)}{|H(u,v)|^2+\gamma |P(u,v)|^2}\right) G(u,v)
\end{equation}

%
%Регуляризация по Тихонову
%
\item Регуляризация по Тихонову. Этот метод также как и фильтр Винера получает оптимальный результат для каждого изображения. Изображение приближается следующим образом:
\begin{equation}\label{iline-shevtsova-eq1}
\hat{F}(u,v) =\left( \frac{1}{H(u,v)} \frac{|H(u,v)|^2}{|H(u,v)|^2+K}\right) G(u,v)
\end{equation}

%
% Люси-Ричардсон
%
\item Итерационный метод Люси-Ричардсона. Этот метод, в отличии от предыдущих, хорошо восстанавливает, когда мы не знаем, какой у нас шум, и на каждой итерации он приближает изображение к исходному.
\begin{equation}\label{iline-shevtsova-eq1}
\hat{f}_{k+1} (x,y) = \hat{f}_k (x,y) \left[ h(x, y) \star \frac{g(x,y}{h(x,y) \otimes \hat{f}_k(x,y)}\right]
\end{equation}

%
%Градиент
%
\item Градиентный спуск. В этом методе минимизируется следующая функция
\begin{equation}\label{iline-shevtsova-eq1}
E = ||g(x,y)-\hat{f}(x,y) \otimes h(x,y)||
\end{equation}

Этот метод, как и метод Люси-Ричардсона итерационный
\begin{equation}\label{iline-shevtsova-eq1}
\frac{\partial E}{\partial \hat{f}_k} =-2 h \star \left(g-\hat{f}_k \otimes h \right)
\end{equation} 
\begin{equation}\label{iline-shevtsova-eq1}
\hat{f}_{k+1} = \hat{f} - \gamma \frac{\partial E}{\partial \hat{f}_k}
\end{equation}
\end{enumerate}

%
%Blind deconvolution
%
\paragraph{Слепые методы}\

К сожалению, ядро размытия почти никогда не известно. В этом случае используются алгоритмы слепой деконволюции. Здесь рассмотрим модификацию метода Люси-Ричардсона
\begin{equation}\label{iline-shevtsova-eq1}
\hat{h}_{k+1} (x,y) = \hat{h}_k (x,y) \left[ \hat{f}_k(x, y) \star \frac{g(x,y}{\hat{h}_k(x,y) \otimes \hat{f}_k(x,y)}\right]
\end{equation}

\begin{equation}\label{iline-shevtsova-eq1}
\hat{f}_{k+1} (x,y) = \hat{f}_k (x,y) \left[ \hat{h}_{k+1}(x, y) \star \frac{g(x,y}{\hat{h}_{k+1}(x,y) \otimes \hat{f}_k(x,y)}\right]
\end{equation}

Здесь очень важно правильно "угадать" первоначальное ядра, иначе, восстановление не всегда удается.

Для борьбы с затемнением восстановленного изображения, применялась адаптивная гамма-коррекция, в результате которой яркость восстановленного изображения становилась равной яркости размытого.

\paragraph{Вывод}\


Несомненно, неслепые методы деконволюции работают лучше чем слепые, однако ядро размытия на практике зачастую неизвестно. Также планируется рассмотреть градиентный спуск для слепой деконволюции и нейронные сети для решения этой задачи\\

В заключение автор выражает признательность научному руководителю, ассистенту кафедры системного анализа и информационных технологий ИВМиИТ Нигматуллину\,Р.\,Р.


%
%   Список литературы
%
\begin{references}
\Source Гонсалес \,Р. , Вудс\, Р. Цифровая обработка изображений. М.:Техносфера, 2012.

\Source  Панфилова\,К.\,В. Компенсаця линейного смаза цифровых изображений с помощью метода Люси-Ричардсона. ГРАФИКОН'2015. 2015. С. 163--167.

\Source  D.\,A.\,Fish, A.\,M.\,Brinicombe, E.\,R.\,Pike Blind deconvolution by means of the Richardson-Lucy algorithm. J. Opt. Soc. Am. A. Vol. 12, No. 1.  January 1995 С 58--65

\Source David \,S. C.\,Biggs Accelerated Iterative Blind Deconvolution. The Departament of Electrical and Electronic Engineering University of Auckland, New Zealand. December
1998

\end{references}

\end{thesis} % Сам тезис должен быть полностью помещен внутри окружения thesis
